%\cvsection{Publications}
\centering{\color{awesome-darknight}\textbf{\large{Research 
publications}} 
({\color{myblue}\href{https://github.com/allegroCoder/my-papers}{link}})}\vspace{-3mm}
\rule{\textwidth}{0.5pt}\vspace{-1mm}

\begin{cventries}
\vspace{-4mm}
%------------------------------------------------

\cventry
{} % Degree
{} % Institution
{} % Location
{} % Date(s)
{ % Description(s) bullet points
\begin{cvitems}
\item{\textbf{``Efficient Composition of Scenario-based Hardware 
Specifications''}, 
\textit{IET CDT} journal, 2018. [\textbf{F}irst \textbf{A}uthor]}
\item{\textbf{``A Heuristic Algorithm for Deriving Compact Models of Processor 
Instruction Sets''}, \textit{ACSD} conference, 2015. 
[\textbf{F.}\textbf{A.}]\newline}
\textbf{Domains:} EDA, high-level synthesis, reconfigurable 
computing, formal modelling.\newline
\textbf{Summary:} Complex hardware systems can be designed by breaking down
their behaviour into high-level descriptions of constituent scenarios, and then
composing these into an efficient hardware implementation using a form
of high-level synthesis. We propose a new algorithm for composition of
scenarios. Unlike previously published methods, the proposed algorithm supports
composition constraints, which allow the designer to restrict certain aspects of
the composition in order to reuse legacy IP. Furthermore, our implementation is
more scalable and can cope with specifications comprising hundreds of scenarios
at the cost of $\simeq$5\% of area overhead compared to exact (and slow) 
solutions, which cannot always be applied. The proposed algorithm is 
implemented in an open-source EDA tool and compared to existing behavioural 
synthesis techniques.
\end{cvitems}
}

%------------------------------------------------
\vspace{-1mm}\rule{8cm}{0.5pt}\vspace{-3mm}
\cventry
{}
{}
{}
{}
{
\begin{cvitems}
\item{\textbf{``Design and Implementation of Reconfigurable Asynchronous 
Pipelines''}, to be submitted to the \textit{IEEE TCAD} journal. 
[\textbf{F.}\textbf{A.}]}
\item{\textbf{``Reconfigurable Asynchronous Pipelines: From Formal Models to 
Silicon''}, \textit{DATE} conference, 2018.\newline}
\item{\textbf{``Prototyping Resilient Processing Cores in Workcraft''}, 
\textit{Resiliency in Embedded Electronic Systems} workshop, 2017.\newline}
\textbf{Domains:} EDA, asynchronous design, low-power and resilient computing, 
formal modelling.\newline
\textbf{Summary:} Pipelining is a widely used approach for designing
high-throughput computation systems. Pipelines are often designed to be
dynamically reconfigurable to process data items differently depending on their
content and/or to adjust to the application requirements in runtime. While
reconfigurable synchronous pipelines are the de facto standard in industry and
are well supported by EDA tools, reconfigurable asynchronous pipelines have
neither a formal behavioural model nor mature automation support. Thus, we
present a model and open-source EDA tool for the design and verification of 
reconfigurable asynchronous pipelines. We validate the presented approach by
designing and fabricating a test ASIC (TSMC 90nm), that demonstrates the
benefits and costs of dynamic reconfigurability, as well as highlights the
resilience of such pipelines.
\end{cvitems}
}

%------------------------------------------------
\rule{8cm}{0.5pt}\vspace{-3mm}
\cventry
{}
{}
{}
{}
{
\begin{cvitems}
\item{\textbf{``Language and Hardware Acceleration Backend for Graph 
Processing''}, published in the \textit{FDL} conference, 2017. And as chapter 
of the book \textit{Languages, Design Methods, and Tools for Electronic System 
Design}, Springer, 2018.}
\item{\textbf{``Distributed Event-Based Computing''}, published in the 
\textit{ParCo} conference, 2017. And as book chapter of the book 
\textit{Parallel Computing is Everywhere}, IOS Press BV, 2018.\newline}
\textbf{Domains:} Programming languages, reconfigurable and high-performance 
computing, many-core architectures.\newline
\textbf{Summary:} Graphs are important in many applications, however their 
analysis on conventional computer architectures is inefficient, because it 
involves highly irregular access to memory when traversing vertices and edges. 
Thus, we present a methodology for embedding graphs into silicon, where graph 
vertices become finite state machines communicating via the graph edges. With 
this approach many common graph analysis tasks can be performed by propagating 
signals through the physical graph and measuring signal propagation time using 
the on-chip clock distribution network. This eliminates the memory bottleneck 
and allows thousands of vertices to be processed in parallel. Also, we present 
a domain-specific language for graph description and transformation, and 
demonstrate how it can be used to translate application graphs into an FPGA 
board, where they can be analysed 1000x faster than on a conventional computer 
cluster.
\end{cvitems}
}

%------------------------------------------------
\rule{8cm}{0.5pt}\vspace{-3mm}
\cventry
{}
{}
{}
{}
{
\begin{cvitems}
\item{\textbf{``Process Windows''}, ACSD conference, 2017\newline}
\textbf{Domains:} formal modelling, concurrency, inter-process 
communication.\newline
\textbf{Summary:} We describe a method for formally representing the behaviour 
of complex processes by \textit{process windows}. Each window covers a part of 
the system behaviour, i.e. a part of the underlying transition system, and is 
easier to understand and analyse than the complete transition system. Process 
windows can overlap and have shared states and transitions so that the complete 
system behaviour is the union of window behaviours. We demonstrate the 
advantage of such representations when dealing with complex system behaviours, 
and discuss potential applications in circuit design and process mining.
\iffalse
\newline
\hspace{+4mm}As a motivational example we consider the problem of covering
transition systems by marked graphs, or more generally choice-free Petri nets. 
The obtained windows correspond to choice-free behavioural scenarios of the 
system, wherein one window can take over, or wake up, after another window has 
become inactive. The corresponding wake-up conditions and wake-up markings can 
be derived automatically.
\fi
\end{cvitems}
}

\iffalse
%------------------------------------------------
\rule{8cm}{0.5pt}\vspace{-3mm}
\cventry
{}
{}
{}
{}
{
\begin{cvitems}
\item{\textbf{``Prototyping Resilient Processing Cores in Workcraft''}, 
\textit{Resiliency in Embedded Electronic Systems} workshop, 2017.\newline}
\textbf{Domains:} EDA, embedded systems, resilient computing.\newline
\textbf{Abstract:} We present a methodology for the design and fast prototyping 
of processing cores with resilient microarchitecture. The resilience is 
achieved by equipping the core with a family of datapath components optimised  
for different operating modes and a flexible control structure that allows to  
change an instruction implementation at runtime depending on current conditions 
and application requirements. We use asynchronous design techniques to  achieve 
short-term resilience, i.e. survival in extreme environmental conditions, such  
as near-threshold or unstable voltage supply. Long-term resilience is achieved 
through runtime reconfiguration of the processor microarchitecture, which is 
essential for safety-critical applications that cannot be taken offline for  
maintenance, such as biomedical implants. By using formal methods one can 
guarantee the correctness and uninterrupted service during such runtime  
reconfigurations.
\iffalse
The presented methodology is supported by open-source  tool WORKCRAFT, and has 
been validated by fabricating two ASICs: an Intel 8051  processing core and a 
reconfigurable dataflow accelerator. To facilitate fast  prototyping of 
resilient processing cores, we introduce a domain-specific  language for their 
formal specification, software-level simulation and hardware synthesis.
\fi
\end{cvitems}
}
\fi

\end{cventries}