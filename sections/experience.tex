%\cvsection{Experience}
\vspace{+2mm}\centering{\color{awesome-darknight}\textbf{\large{Experience}}}\vspace{-3mm}
\rule{\textwidth}{0.5pt}\vspace{-1mm}

\begin{cventries}
\vspace{-5mm}
%------------------------------------------------

\cventry
{}
{\vspace{-9mm}Embedded system engineer \normalfont{at 
\href{http://www.tiempo-secure.com/}{\color{myblue}Tiempo Secure}, Dec 2018 - 
present}} % Job title
{\vspace{-9mm}\normalcolor Grenoble, France} % Location
{} % Date(s)
{
\begin{cvitems}
\item{Design digital blocks of secure embedded systems in SystemVerilog via 
Communicating Sequential Processes --~blocks are described as sets of 
concurrent processes that communicate via IPC, i.e.~shared memory and message 
passing via channels.}
\item{Responsible of the EDA toolchain (in C++) behind our design process, 
which we use and provide to other companies.}
\item{Write validation testbenches in SystemVerilog, low-level drivers 
in C, and automation scripts in Python, Unix shell, GNU Make.}
\item{Work with Agile and Scrum methodologies in a team of 8 engineers.}
\end{cvitems}
}
\vspace{-5mm}
%------------------------------------------------

\cventry
{}
{\vspace{-9mm}Research associate \normalfont{at 
\href{https://www.ncl.ac.uk/engineering/staff/profile/alessandrode-gennaro.html}{\color{myblue}Newcastle
 University}, Mar 
2014 - Oct 2018}} % Job title
{\vspace{-9mm} \normalcolor Newcastle upon Tyne, UK} % Location
{}
{	
\begin{cvitems}
\item {Led the design of an asynchronous processor prototyped on 
ASIC, and of a custom FPGA-based testbench for its
validation~{\color{myblue}\href{https://github.com/tuura/papers/tree/master/date-2018}{\textbf{[1]}}}.
 The prototype was designed to be resilient to voltage-supply variations, and 
to support dynamic hardware reconfiguration.}
\item {Developed C++
tools~{\color{myblue}\href{https://github.com/tuura/shutters}{\textbf{[2}}}{\color{myblue}\textbf{,}}
{\color{myblue}\href{https://github.com/tuura/scenco}{\textbf{3]}}}
for state machine encoding \& synthesis, integrated in the open-source
\href{https://workcraft.org/}{\color{myblue}{\textbf{\textsc{Workcraft}}}} 
environment. The tools were designed with a focus on scalability -- 
~100$\times$ larger specifications are supported by the existing flows.}
\item {Developed a Haskell 
{\color{myblue}\href{https://github.com/tuura/fantasi/tree/master/doc}{\textbf{tool}}}
for synthesis of dynamically reconfigurable hardware accelerators for 
processing 
graphs~{\color{myblue}\href{https://youtu.be/Z2w0hiHY3Us}{\textbf{[4}}}{\color{myblue}\textbf{,}}
{\color{myblue}\href{https://poets-project.org/publications}{\textbf{5]}}}. The 
methodology is employed by 
{\color{myblue}\href{https://www.etherapeutics.co.uk/news-media/videos/}{\textbf{e-Therapeutics}}},
 which can benefit from 10$^{3}\times$ faster analysis of proteins.}
\item {Investigated on a systematic approach for designing Processor 
Instruction 
Sets~{\color{myblue}\href{https://eprint.ncl.ac.uk/file_store/production/251075/92600BF7-92A0-4B22-897A-01892DDA9E2F.pdf}{\textbf{[6]}}},
 published in the IET CDT journal.}
%\item {Expert in asynchronous design techniques at the model- (based on
%	Petri nets) and hardware- levels (dual-rail, bundled-data).}
\end{cvitems}
}
\vspace{-5mm}
%------------------------------------------------

\cventry
{}%\color{myblue}\href{http://www.xeffe.it/}{Xeffe}}
{\vspace{-9mm}Software developer intern \normalfont{at 
{\color{myblue}\href{http://www.xeffe.it/}{Xeffe}}, Feb 2012 - Jul 2012}}
{\vspace{-9mm} \normalcolor Turin, Italy}
{}
{
\begin{cvitems}
Developed a Java web-application in a team of software engineers. I learnt 
to use a relational database (MySQL + MyBatis), and several Java frameworks for 
front-end and back-end development: Spring, Zkoss and Junit. We used Eclipse as 
IDE.
\end{cvitems}
}

%------------------------------------------------

\end{cventries}
%\vspace{-1.5mm}